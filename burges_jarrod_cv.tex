
\documentclass{res} % Use the res.cls style, the font size can be changed to 11pt or 12pt here

\usepackage{fontspec}

% Set the main font to JetBrains Mono
\setmainfont[
  Path=./JetBrainsMono/,
  Scale=0.85,
  Extension = .ttf,
  UprightFont=*-Regular,
  BoldFont=*-Bold,
  ItalicFont=*-Italic,
  BoldItalicFont=*-BoldItalic
]{JetBrainsMono}

% Force sans and mono families to also use JetBrains Mono
\setsansfont[
  Path=./JetBrainsMono/,
  Scale=0.85,
  Extension = .ttf,
  UprightFont=*-Regular,
  BoldFont=*-Bold,
  ItalicFont=*-Italic,
  BoldItalicFont=*-BoldItalic
]{JetBrainsMono}

\setmonofont[
  Path=./JetBrainsMono/,
  Scale=0.85,
  Extension = .ttf,
  UprightFont=*-Regular,
  BoldFont=*-Bold,
  ItalicFont=*-Italic,
  BoldItalicFont=*-BoldItalic
]{JetBrainsMono}

\usepackage{fancyhdr}
\usepackage[hidelinks]{hyperref}
\hypersetup {colorlinks=true,allcolors=blue}

\usepackage[yyyymmdd]{datetime}
\renewcommand{\dateseparator}{--}
\newcommand\reverselabel[1]{%
  \def\theenumi{}%
  \renewcommand\makelabel{\makebox[\dimexpr\labelwidth-3pt\relax][r]{%
    \the\numexpr#1-\value{enumi}+1\relax}}}%

\newsectionwidth{0pt} % Stops section indenting
\renewcommand{\headrulewidth}{0pt}

\usepackage[headheight=42pt,tmargin=42pt,headsep=5pt,textwidth=7in,bmargin=50pt]{geometry}
\pagestyle{fancy}
\fancyhf{}
\rfoot{\textit{Updated \today}}

\lfoot{\thepage}

\begin{document}

%–––––––––––––––––––––––––––––––––––––––
% PERSONAL DATA
%–––––––––––––––––––––––––––––––––––––––
\name{Jarrod Burges, GIT\\ \\}
\address{\bf Wright-Rieman Laboratories \\610 Taylor Road, Room 142 \\ Dept. of Earth and Planetary Sciences \\ Rutgers, The State University of New Jersey \\ Piscataway, NJ 08854-8066 }
\address{\\\href{mailto:jarrod.burges@rutgers.edu}{jarrod.burges@rutgers.edu} \\ \hfill
\href{http://jarrodburges.com}{jarrodburges.com}}

\begin{resume}
%–––––––––––––––––––––––––––––––––––––––
% RESEARCH INTERESTS
%–––––––––––––––––––––––––––––––––––––––
\section{\centerline{RESEARCH INTERESTS}}
My primary research interests involve exploring the applications of machine learning algorithms to solve complex multivariate problems within the geological and planetary science disciplines. In addition, I am passionate about research data management tool/application workflows to conduct more efficient data management for researchers.

%–––––––––––––––––––––––––––––––––––––––
% EDUCATION
%–––––––––––––––––––––––––––––––––––––––

\section{\centerline{EDUCATION}} 
\vspace{8pt}

{\sl Doctor of Philosophy}, Earth and Planetary Sciences \hfill Expected May 2030 \\
Rutgers University – New Brunswick \\
Thesis Advisor: Dr. Shaunna M. Morrison

{\sl Bachelor of Science}, Computer Science \& Geological Sciences \hfill December 2024 \\
California State University – Fullerton  \\
Thesis Advisor: Dr. Kathryn Metcalf \\
Thesis: GeoCORK: An improved workflow for U-Pb geochronology data management

%–––––––––––––––––––––––––––––––––––––––
% PUBLICATIONS
%–––––––––––––––––––––––––––––––––––––––
\section{\centerline{PUBLICATIONS}} 

\begin{enumerate}
\reverselabel{1}%
\setlength{\itemindent}{-.19in}
  \item Metcalf, K. and \textbf{Burges, J.} (2025). GeoCORK: An improved workflow for U-Pb geochronology data management. (In review)
\end{enumerate}

%–––––––––––––––––––––––––––––––––––––––
% CURRENT PROJECTS
%–––––––––––––––––––––––––––––––––––––––
\section{\centerline{CURRENT PROJECTS}} 

\begin{itemize}
\setlength{\itemindent}{-.19in}
  \item Using REEs and machine learning to determine zircon petrogenesis
  \item Disentangling MORB-OIB mantle reservoir end members through machine learning
  \item GeoCORK: Expanding into other geochronological and geochemical methods
\end{itemize}


%–––––––––––––––––––––––––––––––––––––––
% RESEARCH EXPERIENCE
%–––––––––––––––––––––––––––––––––––––––
\section{\centerline{RESEARCH EXPERIENCE}} 

\textbf{Undergraduate Honors Thesis} - California State University, Fullerton \hfill October 2022–Present\\
\emph{GeoCORK: An improved workflow for geochronology data management}
\begin{itemize}

  \item Full‑stack development in Python, SQLite3, and PyQt6 across 52 tables
  \item Designed automated data import, flexible filtering, and export to detritalPy, DzStats, IsoplotR formats
  \item Implemented custom tagging system and UI for complex geological datasets
\end{itemize}

\textbf{Field Assistant} - Tibet, China \hfill July 2023\\
Assisted four‑week expedition collecting structural data for creation of balanced cross‑sections.

\textbf{GIS Analyst} - CSU Fullerton Police 911 Dispatch Center \hfill February 2023\\
Updated CAD geodatabases and satellite imagery in ArcGIS/ArcMap to improve emergency dispatch accuracy.

%–––––––––––––––––––––––––––––––––––––––
% PROFESSIONAL SERVICE
%–––––––––––––––––––––––––––––––––––––––
\section{\centerline{PROFESSIONAL SERVICE}}

\textbf{Student Representative} \hfill October 2023-Present \\ Geological Society of America - Division of Geoinformatics \& Data Science (GIDS)

%–––––––––––––––––––––––––––––––––––––––
% TEACHING \\& ADVISING
%–––––––––––––––––––––––––––––––––––––––
\section{\centerline{TEACHING \& ADVISING}}

\textbf{Teaching Assistant} - 01:460:201:90-91 Earthquakes \& Volcanoes, Rutgers \hfill Fall 2025\\
\textbf{Undergraduate Teaching Assistant} - GEOL303A: Earth Materials, CSU Fullerton \hfill Fall 2023\\
Provided lab instruction for 25 students and delivered a seminar on AI applications in geology. Acted as in-class mentor to sophomore and junior-standing geology students. 

%–––––––––––––––––––––––––––––––––––––––
% PROFESSIONAL EXPERIENCE
%–––––––––––––––––––––––––––––––––––––––
\section{\centerline{PROFESSIONAL EXPERIENCE}}

\textbf{Staff Geologist \& Database Analyst} - EEC Environmental, Orange,CA \hfill May 2024–Present
\begin{itemize}
  \item Authored geological reports and managed multi‑state environmental compliance projects
  \item Built and maintained Microsoft Access/SQL Server databases with full metadata documentation
  \item Monitored drilling operations and provided real‑time geological interpretations
  \item Ensured compliance with environmental regulations across multiple states and company standards during fieldwork and data analysis. 
\end{itemize}

\textbf{Community Service Officer (CSO) Supervisor} - CSU Fullerton Police Dept. \hfill June 2022–March 2025\\
Overseeing supervisor for 35 student CSOs. Managed scheduling, payroll, and field training to ensure program efficiency.

\textbf{Lab Assistant (Winter Temp)} - GMU Geotechnical, Rancho Santa Margarita,CA \hfill December 2022–January 2023
\begin{itemize}
    \item Conducted comprehensive soil analyses to state and federally regulated procedures
\end{itemize}

%–––––––––––––––––––––––––––––––––––––––
% SELECTED AWARDS
%–––––––––––––––––––––––––––––––––––––––
\section{\centerline{SELECTED AWARDS}}

\begin{itemize}
\setlength{\itemindent}{-.25in}
  \item Natural Sciences and Mathematics Inter-Club Council Travel Award (\$750), California State University,Fullerton (November 2024)
  \item Armstrong-Butcher Undergraduate Geology Conference Travel Award (\$500), California State University, Fullerton (November 2024)
  \item Undergraduate Research Opportunity Center (UROC) Travel Award (\$500), California State University, Fullerton (November 2024) 
  \item Armstrong-Butcher Undergraduate Geology Conference Travel Award (\$500), California State University, Fullerton (September 2024) 
  \item Cordilleran Section Student Travel Award (\$375), Geological Society of America (September 2024)
  \item Armstrong-Butcher Undergraduate Geology Conference Travel Award (\$500), California State University, Fullerton (May 2024)
  \item Undergraduate Research Opportunity Center (UROC) Travel Award (\$750), California State  University, Fullerton (May 2024) 
  \item Cordilleran Section Student Travel Award (\$600), Geological Society of America (May 2024)
\end{itemize}

%–––––––––––––––––––––––––––––––––––––––
% ORAL MEETING ABSTRACTS (selected)
%–––––––––––––––––––––––––––––––––––––––

%----------
% 2026
%----------

% MSA Conference Feb 2026
% IMA Conference Feb 2026
% GSA Conference Oct 2026
% AGU Conference Dec 2026

\section{\centerline{ORAL MEETING ABSTRACTS}}

\begin{enumerate}
\reverselabel{6}%
\setlength{\itemindent}{-.25in}

%----------
% 2025
%----------
\item Earth Science Information Partners and Geological Society of America, 2025, Data Help Desk @ GSA Connects 2025:, doi:10.5281/ZENODO.17458264.

\item \textbf{Burges, J.} and Metcalf, K., 2025, GeoCORK: A Desktop Application to Manage U-Pb Geochronology Data: 
Abstract 24 T34-12 presented at 2025 Geological Society of America Annual Meeting, San Antonio, TX, 19-22 October.

\item \textbf{Burges, J.} and Metcalf, K., 2025, GeoCORK: Integrating Existing Databases: 
Abstract 287 T52-2 presented at 2025 Geological Society of America Annual Meeting, San Antonio, TX, 19-22 October.

\item \textbf{Burges, J.} and Metcalf, K., 2024, DESKTOP APPLICATION TO MANAGE AND STORE DETRITAL ZIRCON
GEOCHRONOLOGICAL DATA IN A SQL DATABASE: Abstract 59-4 presented at 2024 Geological Society of
America Annual Meeting, Anaheim, California, 22-25 September.

\item \textbf{Burges, J.} and Metcalf, K., 2024, DESKTOP APPLICATION TO MANAGE AND STORE DETRITAL ZIRCON
GEOCHRONOLOGICAL DATA IN A SQL DATABASE: Abstract 38-2 presented at 2024 Geological Society of
America Joint Cordilleran and Rocky Mountain Section Meeting, Spokane, Washington, 15-17 May.

\item \textbf{Burges, J.}, and Metcalf, K., 2024, DESKTOP APPLICATION TO MANAGE AND STORE
DETRITAL ZIRCON GEOCHRONOLOGICAL DATA IN A SQL DATABASE: Abstract
411 presented at 2024 National Conference of Undergraduate Research, Long Beach, California, 8-10 April.
\end{enumerate}

\section{\centerline{POSTER MEETING ABSTRACTS}}

\begin{enumerate}
\reverselabel{8}%
\setlength{\itemindent}{-.25in}

%----------
% 2024
%----------
\item \textbf{Burges, J.} and Metcalf, K., 2024, DESKTOP APPLICATION TO MANAGE AND STORE DETRITAL ZIRCON
GEOCHRONOLOGICAL DATA IN A SQL DATABASE: Abstract V23B-3335 presented at 2024 American 
Geophysical Union Annual Meeting, Washington, DC, 9-13 December.

\item \textbf{Burges, J.} and Metcalf, K., 2024, DESKTOP APPLICATION TO MANAGE AND STORE DETRITAL ZIRCON
GEOCHRONOLOGICAL DATA IN A SQL DATABASE: Abstract 38-2 presented at 2024 Geological Society of
America Joint Cordilleran and Rocky Mountain Section Meeting, Spokane, Washington, 15-17 May.
\item \textbf{Burges, J.} and Metcalf, K., 2024, DESKTOP APPLICATION TO MANAGE AND STORE DETRITAL ZIRCON
GEOCHRONOLOGICAL DATA IN A SQL DATABASE: Presented at Southern California Geological Society
May 2024 Meeting, Fullerton, California, 6 May.
\item \textbf{Burges, J.} and Metcalf, K., 2024, DESKTOP APPLICATION TO MANAGE AND STORE DETRITAL ZIRCON
GEOCHRONOLOGICAL DATA IN A SQL DATABASE: Presented at 2024 Department of Geological Sciences
Research Day, Fullerton, California, 3 May.

%----------
% 2023
%----------
\item \textbf{Burges, J.}, and Metcalf, K., 2023, DESKTOP APPLICATION TO MANAGE AND STORE
DETRITAL ZIRCON GEOCHRONOLOGICAL DATA IN A SQL DATABASE: Abstract
46-6 presented at 2023 Geological Society of America Annual Meeting, Pittsburgh, Philadelphia, 15-18
October.
\item \textbf{Burges, J.}, Metcalf, K., and Goffman, M., 2023, PYTHON PROGRAM TO INPUT, SORT, VIEW, AND STORE
DETRITAL ZIRCON GEOCHRONOLOGICAL DATA IN A SQL DATABASE: Abstract 9-11 presented at 2023
Geological Society of America Cordilleran Section Meeting, Reno, Nevada, 16-19 May.
\item \textbf{Burges, J.}, Metcalf, K., and Goffman, M., 2023, PYTHON PROGRAM TO INPUT, SORT, VIEW, AND STORE
DETRITAL ZIRCON GEOCHRONOLOGICAL DATA IN A SQL DATABASE: Presented at 2023 College of
Engineering and Computer Science Student Innovation Expo, Fullerton, California, 5 May.
\item \textbf{Burges, J.}, Metcalf, K., and Goffman, M., PYTHON PROGRAM TO INPUT, SORT, VIEW, AND STORE DETRITAL
ZIRCON GEOCHRONOLOGICAL DATA IN A SQL DATABASE: Presented at 2023 Department of Geological
Sciences Research Day, Fullerton, California, 5 May
\end{enumerate}

%–––––––––––––––––––––––––––––––––––––––
% CERTIFICATIONS
%–––––––––––––––––––––––––––––––––––––––
\section{\centerline{CERTIFICATIONS}} 

\begin{itemize}
\setlength{\itemindent}{-.25in}
    \item California Geologist‑in‑Training (GIT) - Obtained October 2024
\end{itemize}

%–––––––––––––––––––––––––––––––––––––––
% TECHNICAL SKILLS
%–––––––––––––––––––––––––––––––––––––––
\section{\centerline{TECHNICAL SKILLS}} 

\begin{itemize}
\setlength{\itemindent}{-.25in}
  \item \textbf{Languages}: Python, SQL, R, Java, C++
  \item \textbf{Development}: JetBrains IDE Suite, RStudio, Git/GitHub
  \item \textbf{GIS}: ArcGIS Pro / Toolbox
  \item \textbf{Productivity}: Microsoft Office Suite
  \item \textbf{Design}: Adobe Illustrator, Acrobat Sign, Photoshop, Premiere Pro
  \item \textbf{OS}: Linux(Ubuntu / Raspbian), Windows, Mac OS
\end{itemize}

%–––––––––––––––––––––––––––––––––––––––
% REFERENCES
%–––––––––––––––––––––––––––––––––––––––
\section{\centerline{References}}

Available upon request for both academic and/or professional.

\end{resume}
\end{document}
