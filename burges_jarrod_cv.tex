
\documentclass[10pt]{res} % Use the res.cls style, the font size can be changed to 11pt or 12pt here

\usepackage{helvet} % Default font is the helvetica postscript font
%\usepackage{newcent} % To change the default font to the new century schoolbook postscript font uncomment this line and comment the one above
\usepackage{fancyhdr}
\usepackage{lastpage}

\usepackage[yyyymmdd]{datetime}
\renewcommand{\dateseparator}{--}

\newsectionwidth{0pt} % Stops section indenting
\renewcommand{\headrulewidth}{0pt}

\usepackage[headheight=42pt,tmargin=42pt,headsep=5pt,textwidth=7in,footheight=0pt,bmargin=50pt,footsep=6pt]{geometry}
\pagestyle{fancy}
\fancyhf{}
\rfoot{\textit{{Updated \today}}}

\lfoot{\thepage}

\begin{document}

%–––––––––––––––––––––––––––––––––––––––
% PERSONAL DATA
%–––––––––––––––––––––––––––––––––––––––
\name{Jarrod Burges, GIT\\ \\}
\address{{\bf Wright-Rieman Laboratories \\610 Taylor Road, Room 142} \\ Dept. of Earth and Planetary Sciences \\ Rutgers, The State University of New Jersey \\ Piscataway, NJ 08854-8066 }
\address{{\bf Permanent Address} \\ 123 Broadway \\ City, State 12345 \\ (000) 111-1111} % Your address 2



\begin{resume}
%–––––––––––––––––––––––––––––––––––––––
% RESEARCH INTERESTS
%–––––––––––––––––––––––––––––––––––––––
\section{\centerline{Research Interests}}
Utilizing machine learning to solve geological questions using big data, with specific interests in developing models to understand geochemical and optical–petrographic micro‑signatures. Passionate about building geoscience cyberinfrastructure to accelerate data analysis.

%–––––––––––––––––––––––––––––––––––––––
% EDUCATION
%–––––––––––––––––––––––––––––––––––––––

\section{\centerline{EDUCATION}} 
\vspace{8pt}

{\sl Doctor of Philosophy},
Earth and Planetary Sciences \\ Rutgers University – New Brunswick \hfill Expected May 2030 \\
Thesis Advisor: Dr. Shaunna M. Morrison

{\sl Bachelor of Science},
Geological Sciences \\ California State University – Fullerton \hfill December 2024 \\
Thesis Advisor: Dr. Kathryn Metcalf \\
\\ Thesis: GeoCORK: An improved workflow for geochronology data management

%–––––––––––––––––––––––––––––––––––––––
% CERTIFICATIONS
%–––––––––––––––––––––––––––––––––––––––
\section{\centerline{EDUCATION}} 
\vspace{8pt} 

\begin{itemize}
    \item California Geologist‑in‑Training (GIT) — Obtained October 2024
\end{itemize}


%–––––––––––––––––––––––––––––––––––––––
% PUBLICATIONS
%–––––––––––––––––––––––––––––––––––––––
\section{\centerline{PUBLICATIONS}} 
\vspace{8pt} 

\begin{enumerate}
  \item Metcalf,K. and \textbf{Burges,J.} (2024). GeoCORK: Part1. An improved workflow for geochronology data management. (In preparation; submission April2025)
  \item \textbf{Burges,J.} and Metcalf,K. (2024). GeoCORK: Part2 Technical specifications for future software development and expansion. (In preparation; submission April2025)
\end{enumerate}

%–––––––––––––––––––––––––––––––––––––––
% TECHNICAL SKILLS
%–––––––––––––––––––––––––––––––––––––––
\section{\centerline{TECHNICAL SKILLS}} 
\vspace{8pt} 

\begin{itemize}
  \item \textbf{Languages}: Python, SQL, R, Java, C++
  \item \textbf{Development}: JetBrainsIDESuite, RStudio, Git/GitHub
  \item \textbf{GIS}: ArcGIS Pro / ArcMap
  \item \textbf{Productivity}: Microsoft Office Suite
  \item \textbf{Design}: Adobe Illustrator, Acrobat Sign, Photoshop, Premiere Pro
  \item \textbf{OS}: Linux(Ubuntu/Raspbian), Windows, macOS
\end{itemize}

%–––––––––––––––––––––––––––––––––––––––
% RESEARCH EXPERIENCE
%–––––––––––––––––––––––––––––––––––––––
\section{\centerline{RESEARCH EXPERIENCE}} 
\vspace{8pt} 

\textbf{Undergraduate Honors Thesis} — California State University, Fullerton \hfill October 2022–Present\\
\emph{GeoCORK: An improved workflow for geochronology data management}
\begin{itemize}
  \item Full‑stack development in Python, SQLite3, and PyQt6 across 52 normalized tables
  \item Designed automated data import, flexible filtering, and export to detritalPy, DzStats, IsoplotR
  \item Implemented custom tagging system and UI for complex geological datasets
\end{itemize}

\textbf{Field Assistant} — Tibet, China \hfill July 2023\\
Assisted four‑week expedition collecting structural data for creation of balanced cross‑sections.

\textbf{GIS Analyst} — CSU Fullerton Police 911 Dispatch Center \hfill February 2023\\
Updated CAD geodatabases and satellite imagery in ArcGIS/ArcMap to improve emergency dispatch accuracy.

%–––––––––––––––––––––––––––––––––––––––
% TEACHING \\& ADVISING
%–––––––––––––––––––––––––––––––––––––––
\section{\centerline{TEACHING \& ADVISING}}
\vspace{8pt} 

\textbf{Undergraduate Teaching Assistant} — GEOL303A: EarthMaterials, CSUFullerton \hfill Fall 2023\\
Provided lab instruction for 25 students and delivered a seminar on AI applications in geology. Acted as in-class mentor to sophomore and junior-standing geology students. 


%–––––––––––––––––––––––––––––––––––––––
% PROFESSIONAL EXPERIENCE
%–––––––––––––––––––––––––––––––––––––––
\section{\centerline{PROFESSIONAL EXPERIENCE}}
\vspace{8pt} 
\textbf{Staff Geologist \& Database Analyst} — EEC Environmental, Orange,CA \hfill May 2024–Present
\begin{itemize}
  \item Authored geological reports and managed multi‑state environmental compliance projects
  \item Built and maintained MicrosoftAccess/SQLServer databases with full metadata documentation
  \item Monitored drilling operations and provided real‑time geological interpretations
  \item Ensured compliance with environmental regulations across multiple states and company standards during fieldwork and data analysis. 
\end{itemize}

\textbf{Community Service Officer (CSO) Supervisor} — CSU Fullerton Police Dept. \hfill June 2022–March 2025\\
Overseeing supervisor for 35 student CSOs. Managed scheduling, payroll, and field training to ensure program efficiency.

\textbf{Lab Assistant (Winter Temp)} — GMU Geotechnical, Rancho Santa Margarita,CA \hfill December 2022–January 2023
\begin{itemize}
    \item Conducted comprehensive soil analyses to state and federally regulated procedures
\end{itemize}

%–––––––––––––––––––––––––––––––––––––––
% PROFESSIONAL SERVICE
%–––––––––––––––––––––––––––––––––––––––
\section{\centerline{PROFESSIONAL SERVICE}}
\vspace{8pt} 
\textbf{Student Representative} \hfill October 2023-Present \\ Geological Society of America — Division of Geoinformatics \& Data Science (GIDS)

%–––––––––––––––––––––––––––––––––––––––
% SELECTED AWARDS
%–––––––––––––––––––––––––––––––––––––––
\section{\centerline{SELECTED AWARDS}}
\vspace{16pt} 
\begin{itemize}
  \item Natural Sciences and Mathematics Inter-Club Council Travel Award (\$750), California State University,Fullerton (November 2024)
  \item Armstrong-Butcher Undergraduate Geology Conference Travel Award (\$500), California State University, Fullerton (November 2024)
  \item Undergraduate Research Opportunity Center (UROC) Travel Award (\$500), California State University, Fullerton (November 2024) 
  \item Armstrong-Butcher Undergraduate Geology Conference Travel Award (\$500), California State University, Fullerton (September 2024) 
  \item Cordilleran Section Student Travel Award (\$375), Geological Society of America (September 2024)
  \item Armstrong-Butcher Undergraduate Geology Conference Travel Award (\$500), California State University, Fullerton (May 2024)
  \item Undergraduate Research Opportunity Center (UROC) Travel Award (\$750), California State  University, Fullerton (May 2024) 
  \item Cordilleran Section Student Travel Award (\$600), Geological Society of America (May 2024)
\end{itemize}

%–––––––––––––––––––––––––––––––––––––––
% MEETING ABSTRACTS (selected)
%–––––––––––––––––––––––––––––––––––––––
\section{\centerline{SELECT AWARDS}}
\vspace{16pt} 
\begin{enumerate}
  \item \textbf{Burges,J.} and Metcalf, K., 2024, DESKTOP APPLICATION TO MANAGE AND STORE DETRITAL ZIRCON
GEOCHRONOLOGICAL DATA IN A SQL DATABASE: Abstract V23B-3335 presented at 2024 American
Geophysical Union Annual Meeting, Washington, DC, 9-13 December.
April 2025 Page 5 of 5
\item \textbf{Burges,J.} and Metcalf, K., 2024, DESKTOP APPLICATION TO MANAGE AND STORE DETRITAL ZIRCON
GEOCHRONOLOGICAL DATA IN A SQL DATABASE: Abstract 59-4 presented at 2024 Geological Society of
America Annual Meeting, Anaheim, California, 22-25 September.
\item \textbf{Burges,J.} and Metcalf, K., 2024, DESKTOP APPLICATION TO MANAGE AND STORE DETRITAL ZIRCON
GEOCHRONOLOGICAL DATA IN A SQL DATABASE: Abstract 38-2 presented at 2024 Geological Society of
America Joint Cordilleran and Rocky Mountain Section Meeting, Spokane, Washington, 15-17 May.
\item \textbf{Burges,J.} and Metcalf, K., 2024, DESKTOP APPLICATION TO MANAGE AND STORE DETRITAL ZIRCON
GEOCHRONOLOGICAL DATA IN A SQL DATABASE: Presented at Southern California Geological Society
May 2024 Meeting, Fullerton, California, 6 May.
\item \textbf{Burges,J.} and Metcalf, K., 2024, DESKTOP APPLICATION TO MANAGE AND STORE DETRITAL ZIRCON
GEOCHRONOLOGICAL DATA IN A SQL DATABASE: Presented at 2024 Department of Geological Sciences
Research Day, Fullerton, California, 3 May.
\item \textbf{Burges,J.}, and Metcalf, K., 2024, DESKTOP APPLICATION TO MANAGE AND STORE
DETRITAL ZIRCON GEOCHRONOLOGICAL DATA IN A SQL DATABASE: Abstract
411 presented at 2024 National Conference of Undergraduate Research, Long Beach, California, 8-10 April.
\item \textbf{Burges,J.}, and Metcalf, K., 2023, DESKTOP APPLICATION TO MANAGE AND STORE
DETRITAL ZIRCON GEOCHRONOLOGICAL DATA IN A SQL DATABASE: Abstract
46-6 presented at 2023 Geological Society of America Annual Meeting, Pittsburgh, Philadelphia, 15-18
October.
\item \textbf{Burges,J.}, Metcalf, K., and Goffman, M., 2023, PYTHON PROGRAM TO INPUT, SORT, VIEW, AND STORE
DETRITAL ZIRCON GEOCHRONOLOGICAL DATA IN A SQL DATABASE: Abstract 9-11 presented at 2023
Geological Society of America Cordilleran Section Meeting, Reno, Nevada, 16-19 May.
\item \textbf{Burges,J.}, Metcalf, K., and Goffman, M., 2023, PYTHON PROGRAM TO INPUT, SORT, VIEW, AND STORE
DETRITAL ZIRCON GEOCHRONOLOGICAL DATA IN A SQL DATABASE: Presented at 2023 College of
Engineering and Computer Science Student Innovation Expo, Fullerton, California, 5 May.
\item \textbf{Burges,J.}, Metcalf, K., and Goffman, M., PYTHON PROGRAM TO INPUT, SORT, VIEW, AND STORE DETRITAL
ZIRCON GEOCHRONOLOGICAL DATA IN A SQL DATABASE: Presented at 2023 Department of Geological
Sciences Research Day, Fullerton, California, 5 May
\end{enumerate}

%–––––––––––––––––––––––––––––––––––––––
% REFERENCES
%–––––––––––––––––––––––––––––––––––––––
\section{\centerline{References}}
\vspace{8pt} 
Available upon request.

\end{document}
